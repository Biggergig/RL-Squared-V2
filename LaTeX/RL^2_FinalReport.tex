\documentclass[conference]{IEEEtran}
\usepackage[utf8]{inputenc}
\usepackage{amsmath}
\usepackage{amsfonts}
\usepackage{amssymb}
\usepackage{graphicx}
\usepackage{hyperref}
\usepackage{bookmark}
\usepackage{algorithm}
\usepackage{algpseudocode}
\usepackage{cite}

\title{RL-Squared Final Report}
\author{
    \IEEEauthorblockN{Anshuman Dash, Phillip Peng, Matthew Shen}
    \IEEEauthorblockA{
        Department of Computer Science\\
        University of California, Santa Barbara\\
        Email: \{anshumandash, phillippeng, matthewshen\}@ucsb.edu
    }
}
\date{Winter 2025}


\begin{document}

\maketitle


\begin{abstract}
We aim to investigate the use of Reinforcement Learning in the context of the game of Rocket League. Through the uses of PPO and Curriculum Learning through self-play, we train bots on a variety of reward structures and hyperparameters. To do an empirical comparison of the different models, we evaluate the bots against each other in a tournament setting of 1v1 matches to determine model ELO. We analyze the results of this tournament as well as individual models and discuss the implications of our findings.
\end{abstract}

\section{Introduction}
Reinforcement Learning (RL) is a powerful tool for training agents to perform tasks in a variety of environments. We aim to investigate the use of RL in the context of the game of Rocket League. Rocket League is a popular video game that combines soccer with rocket-powered cars. The game is played in a 3D environment, and players control their cars to hit a ball into the opposing team's goal. The game itself holds a large number of states making it a challenging environment for RL agents to learn in. 

\section{Related Work}
The use of RL to play games has been a popular research topic in recent years, most famously with AlphaGo. The application of RL in other games such as Dota 2  and Starcraft II  has also been explored. 

\subsection{PPO}

\begin{algorithm*}
    \caption{PPO Algorithm}
    \begin{algorithmic}[1]
        \State \textbf{Input:} initial policy parameters $\theta_0$, initial value function parameters $\phi_0$
        \For {k = 0,1,2, ... } 
            \State Collect set of trajectories $\mathcal{D}_k$ by running policy $\pi_k = \pi(\theta_k)$ in the environment
            \State Compute rewards-to-go $\hat{R}_t$ 
            \State Compute advantage estimates $\hat{A}_t$ for current value function $V_{\phi_k}$
            \State Update policy by maximizing PPO-Clip objective:
            \State \hspace{1cm} $\theta_{k+1} = \arg\underset{\theta}{\max}{\theta} \mathbb{E}_t \left[ \frac{\pi_{\theta}(a_t|s_t)}{\pi_{\theta_k}(a_t|s_t)} \hat{A}_t \right]$
        \EndFor
    \end{algorithmic}
    \label{alg:PPO}
\end{algorithm*}

\section{Model and Methods}
Through this work we largely explore how a change in reward structure as well as model hyperparameters affect the performance of trained agents. All models are trained using the Proximal Policy Optimization (PPO) algorithm.

\subsection{Reward Scaling}

\subsection{Reward Structures}

\subsection{Curriculum Learning}
Curriculum learning is a training strategy where the learning process is organized in a way that gradually increases in complexity. 

We find that this doesn't perform well for a variety of reasons. The increasing sparsification of rewards makes it difficult for the agents to learn effectively. In order to address this, we propose a modified approach that balances exploration and exploitation.

\section{Results}
We evaluate the performance of the trained agents by having them play against each other in a Placket Luce tournament \cite{ranking2010label} 


\section{Conclusion}

\section{Future Work}

\bibliographystyle{IEEEtran}
\bibliography{references}

\end{document}

